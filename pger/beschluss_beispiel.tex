\documentclass[notitlepage]{scrartcl}
\usepackage{pgerde}

\draft{1}
\anontrue
\name{hans}{Hans Muster}{A}
\name{fritz}{Fritz Meier}{B}

\decision
\casenumber{AS.2.10}
\decisiondate{1. Dezember 2013}
\judges{Richter Stefan Thöni, Präsident \newline
		als Einzelschiedsrichter der Piratenpartei Schweiz \newline
		mit Sitz in Bern}
\plaintiff{\textbf{\anonhans}}
\respondent{\textbf{\anonfritz}}
\subject{Parteiausschluss}

\begin{document}

\maketitle

\newpage

\N{Konstituierung}{\lipsum[10]}

\N{Anträge}{}

\NN{\anonhans}{\lipsum[12]}

\N{Sachverhalt}{\lipsum[17]}

\N{Eintreten}{\lipsum[18]}

\newpage
\N{Dispositiv}{
\textbf{Das Piratengericht erkennt:}

\begin{minipage}{12cm}
\begin{enumerate}
\item Auf die Anträge wird nicht eingetreten.
\item Es werden keine Verfahrenskosten auferlegt.
\end{enumerate}
\end{minipage}

\vspace{0.2cm}
Dieses Urteil wird mitgeteilt:

\begin{minipage}{12cm}
\begin{enumerate}
\item \anonfritz
\item \anonhans
\end{enumerate}
\end{minipage}
}

\signature{Stefan Thöni, Präsident}

\appeal

\end{document}