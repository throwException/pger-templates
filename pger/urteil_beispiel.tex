\documentclass[notitlepage,nonpublic]{scrartcl}
\usepackage{pgerde}

%\draft{1}
%\makeanon
\pseudonym{HM}{Hans Muster}{A}
\anonym{HMa}{Musterweg 1337a, 9999 Hintertupfingen}
\pseudonym{NM}{Nina Meyer}{A}
\anonym{NMa}{Bahnhofstr. 42, 9999 Hintertupfingen}
\anonym{JAa}{Gemeinweg 1a, 1001 Ostermundigen}

\judgement
\division{I.}
\casenumber{1E.23/2018}
\decisiondate{1. Dezember 2013}
\judges{\APp \newline
		\FHr \newline
		\FRr}
\plaintiff{%
\actor{}{Piratenpartei Schweiz}{3000 Bern}
\actor{handelnd durch Vizepräsident}{Jorgo Ananiadis}{\JAa}
Klägerin,}
\respondent{%
\actor{Herr}{\HM}{\HMa}
\actor{vertreten durch}{\NM}{\NMa}
Beklagter,}
\subject{Parteiausschluss.}

\begin{document}

\maketitle

\newpage

\begin{numbering}

\N{Konstituierung}{\lipsum[10]}

\N{Anträge}{}

\NN{Vorstand}{\lipsum[12]}

\NN{\HM}{\lipsum[13]}

\N{Sachverhalt}{}

\nn{\lipsum[14-15]}

\nn{\lipsum[16]}

\nn{\lipsum[17]}

\N{Eintreten}{\lipsum[18]}

\N{Erwägungen}{}

\nn{\lipsum[19]}

\nn{\lipsum[20-21]}

\nn{\lipsum[22]}

\nn{Gemäss \zgbf[11] und \zgb[11][1] sowie \zgb[11][1][a] ist das so, weil ist so.}

\nn{Gemäss \stgbf[48][][][a][1], \stgb[303][1][1] und \zgb[178][1][][1] ist das so, weil ist so.}

\nn{\nonpublic{Dieser Text ist nicht öffentlich!}}

\newpage

\N{Dispositiv}

\textbf{Das Piratengericht erkennt:}

\begin{enumerate}
\item Die Anträge werden abgewiesen, soweit darauf eingetreten wird.
\item{Die Verfahrenskosten betragen: \vspace{0.3cm} \\
\vspace{0.3cm}
\begin{tabular}{l r r}
Miete Gerichtssaal \hspace{1cm} & \hspace{1cm} Fr. & \hspace{1cm} 60.00 \\
4 Einschreiben & Fr. & 24.00 \\
\hline
Total & Fr. & 84.00 \\
\end{tabular} \\
Die Verfahrenskosten werden dem Vorstand auferlegt.}
\end{enumerate}

\vspace{0.2cm}
Dieses Urteil wird mitgeteilt:

\begin{enumerate}
\item \HM
\item dem Vorstand der Piratenpartei Schweiz
\end{enumerate}

\end{numbering}

\signature{Stefan Thöni, Präsident}

\appeal

\end{document}