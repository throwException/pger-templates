\documentclass[notitlepage]{scrartcl}
\usepackage{pgerde}

\draft{1}
\anontrue
\name{hans}{Hans Muster}{A}

\judgement
\casenumber{AS.2013.9}
\decisiondate{1. Dezember 2013}
\judges{Richter Stefan Thöni, Präsident, Sabrina Andali und Florian Mauchle, parteiberufene Richter Fritz Meier und Anna Müller}
\plaintiff{\textbf{VORSTAND DER PIRATENPARTEI SCHWEIZ}, vertreten durch Vizepräsident Pascal Gloor}
\respondent{\textbf{\anonhans}}
\subject{Parteiausschluss}

\begin{document}

\maketitle

\newpage

\N{Konstituierung}{\lipsum[10]}

\N{Anträge}{}

\NN{Vorstand}{\lipsum[12]}

\NN{\anonhans}{\lipsum[13]}

\N{Sachverhalt}{}

\nn{\lipsum[14-15]}

\nn{\lipsum[16]}

\nn{\lipsum[17]}

\N{Eintreten}{\lipsum[18]}

\N{Erwägungen}{}

\nn{\lipsum[19]}

\nn{\lipsum[20-21]}

\nn{\lipsum[22]}

\nn{\lipsum[23]}

\newpage
\N{Dispositiv}{
\textbf{Das Piratengericht erkennt:}

\begin{minipage}{12cm}
\begin{enumerate}
\item Die Anträge werden abgewiesen, soweit darauf eingetreten wird.
\item{Die Verfahrenskosten betragen: \vspace{0.3cm} \\
\vspace{0.3cm}
\begin{tabular}{l r r}
Miete Gerichtssaal \hspace{1cm} & \hspace{1cm} Fr. & \hspace{1cm} 60.00 \\
4 Einschreiben & Fr. & 24.00 \\
\hline
Total & Fr. & 84.00 \\
\end{tabular} \\
Die Verfahrenskosten werden dem Vorstand auferlegt.}
\end{enumerate}
\end{minipage}

\vspace{0.2cm}
Dieses Urteil wird mitgeteilt:

\begin{minipage}{12cm}
\begin{enumerate}
\item \anonhans
\item dem Vorstand der Piratenpartei Schweiz
\end{enumerate}
\end{minipage}
}

\signature{Stefan Thöni, Präsident}

\appeal

\end{document}